\documentclass[12pt, a4paper]{article}

\usepackage[ngerman]{babel} 
\usepackage[T1]{fontenc}
\usepackage{amsfonts} 
\usepackage{setspace}
\usepackage{amsmath}
\usepackage{amssymb}
\usepackage{titling}
\usepackage{csquotes} % for \textquote{}
\usepackage{hyperref}
\usepackage{tikz}
\usepackage{stix}
\usepackage{stmaryrd} % \llbracket and \rrbracket
\usetikzlibrary{arrows, automata, positioning}

\newcommand*{\qed}{\null\nobreak\hfill\ensuremath{\square}}
\newcommand*{\lqed}{\null\nobreak\hfill\ensuremath{\blacksquare}}
\newcommand*{\puffer}{\text{ }\text{ }\text{ }\text{ }}
\newcommand*{\gedanke}{\textbf{-- }}
\newcommand*{\gap}{\text{ }}
\newcommand*{\setDef}{\gap|\gap}
\newcommand*{\vor}{\textbf{Vor.:} \gap}
\newcommand*{\beh}{\textbf{Beh.:} \gap}
\newcommand*{\bew}{\textbf{Bew.:} \gap}
% Hab länger gebraucht um zu realisieren, dass das ne gute Idee wäre
\newcommand*{\R}{\mathbb R}
% Das hier wird sehr nett werden
\newcommand*{\vDashR}{\mathrel{\reflectbox{\(\vDash\)}}}
\newcommand*{\LeftrightvDash}{\vDash\vDashR}
\newcommand{\lb}{\llbracket}
\newcommand{\rb}{\rrbracket}
\newcommand{\rbb}{\rrbracket_{\beta}}

\newenvironment{noalign*}
 {\setlength{\abovedisplayskip}{0pt}\setlength{\belowdisplayskip}{0pt}%
  \csname flalign*\endcsname}
 {\csname endflalign*\endcsname\ignorespacesafterend}



\pagestyle{plain}
\allowdisplaybreaks

\setlength{\droptitle}{-11em}
\setlength{\jot}{12pt}
%\setlength{\hoffset}{-1in}     Wenn nötig
%\setlength{\textwidth}{535pt}  Wenn nötig

\title{Operating Systems\\Assignment 5.1 - Content Questions}
\author{Henri Heyden, Nike Pulow \\ \small stu240825, stu239549}
\date{}


\begin{document}
\maketitle

\singlespacing
\vspace*{-2cm}
\subsection*{5.1.1}
\subsubsection*{Why can we not simply use time or clocks as means for synchronization?}
Synchronization via time would usually result in slower processes, as for every process there must 
be a specific amount of time specified, in which this process needs to complete their task. This would 
also mean that if the process finishes its task way before, it still would have to wait for the timeout 
to occur. On the other hand, if a process needs more time than specified in its timeout interval, this 
might result in processes never being able to finish. \\
The same line of arguments goes for clocks - clocks are a great way to synchronize with real world time; 
yet, as mentioned above, time is not a great way for synchronization. Logical clocks on the other hand provide 
some control over the order of events in a distributed system or pose as useful checkpoints in alogrithms. \\
In summery: semaphores provide more controlled access to critical sections and shared resources, which of 
course is the whole point of synchronization in this context. While time might be able to do the same thing, 
it is just way more efficient to use semaphores, as there is less waste of resources because of waiting times.

\subsection*{5.1.2}
\subsubsection*{What are deadlocks?}
''Deadlock is the situation in which one or more processes within a system are blocked forever due to 
requirements which can never be satisfied.'' \\(Holt, 1971, p. 1)\\ In other words: In a deadlock situation 
a process A is waiting for a process B to release a resource, while process B is waiting on process A to 
release a resource.\\
\subsubsection*{What are livelocks?}
A livelock occurs when ''one or more threads continuously change their states in response to changes in 
states of the other threads without doing any useful work.'' (Ganai, 2013, p. 1)\\
\subsubsection*{What is the difference and are they both equally difficult to resolve?}
The main difference lies in what causes a live-/deadlock to occur: for deadlocks it is dependencies, for 
livelocks it is a status change in response to another status change. \\
Livelocks are harder to recognise than deadlocks, since the program counter of the involved processes 
is constantly changing, due to the status changes of the processes. There are several different strategies 
to solve both problems, all of which are dependent on circumstance. Usually, the deadlock problem is supposed 
to be handled by the application, whereas prevention is practised in real-time operation, while detection is 
the way to go in debugging situations.\\
Livelocks are usually being resolved by changing the decision-making logic (refer to the philosopher's problem) 
or by introducing randomness and delays. \\
If detected and analyzed properly, both problems are equally difficult to solve, just in different ways. 

\subsection*{5.1.3}
\subsubsection*{Can you use a single semaphore as a lock? More precisely, can every use-case of a lock be 
implemented by using a single semaphore? Why?}
While it is genereally possible to implement a lock by using a single (binary) semaphore, it may not be 
the best idea, as mutexes for example serve other purposes as well as provide additional functionality.

\subsubsection*{Can you directly use a single lock as a semaphore? More precisely, can every use-case of 
a semaphore be implemented by using a single lock? Why?}
Generally, this is not possible. There may be use-cases of binary semiphores that are implemented by using a 
lock, but as soon as there is more than one resource managed by a (counting) semaphore, it will not be 
possible to efficiently implement it using a single lock. The main problem in this scenario is, that a single 
exclusive lock will only be able to manage one resource, so either all the other resources remain unmanaged, 
leading to a whole other world of problems, or all other resources remain unused, which is horrible from 
a performance point of view.

\subsubsection*{Does this mean they are equivalent or is one of them the more powerful abstraction?}
There are different use-cases, but generally, speaking, what can be done with locks might also be done with 
semaphores, and more (refer to ring buffers). As semaphores were introduced by Dijkstra in 1968 to solve problems 
that could not have been solved using locks (e. g. ''busy wait''), it is save to say, that semaphores indeed 
are the more powerful abstraction.

\subsection*{5.1.4}
\subsubsection*{What is RPC?}
\subsubsection*{What is the idea of RPC?}
\subsubsection*{What is a stub in the context of RPC?}
\subsubsection*{What is a skeleton in the context of RPC?}
\end{document}