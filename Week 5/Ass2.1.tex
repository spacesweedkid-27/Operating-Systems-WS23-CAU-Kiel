\documentclass[12pt, a4paper]{article}

\usepackage[english]{babel} 
\usepackage[T1]{fontenc}
\usepackage{amsfonts} 
\usepackage{setspace}
\usepackage{amsmath}
\usepackage{amssymb}
\usepackage{titling}

\usepackage[
    left = \glqq{},% 
    right = \grqq{},% 
    leftsub = \glq{},% 
    rightsub = \grq{} %
]{dirtytalk}

\newcommand*{\qed}{\null\nobreak\hfill\ensuremath{\square}}
\newcommand*{\puffer}{\text{ }\text{ }\text{ }\text{ }}
\newcommand*{\gedanke}{\textbf{-- }}
\newcommand*{\gap}{\text{ }}
\newcommand*{\setDef}{\gap|\gap}
\newcommand*{\vor}{\textbf{Vor.:} \gap}
\newcommand*{\beh}{\textbf{Beh.:} \gap}
\newcommand*{\bew}{\textbf{Bew.:} \gap}
% Hab länger gebraucht um zu realisieren, dass das ne gute Idee wäre
\newcommand*{\R}{\mathbb R}


\pagestyle{plain}
\allowdisplaybreaks

\setlength{\droptitle}{-14em}
\setlength{\jot}{12pt}

\title{Operating Systems\\Assignment 2.1}
\author{Henri Heyden, Nike Pulow \\ \small stu240825, stu239549}
\date{}


\begin{document}
\maketitle

\singlespacing

\subsection*{1)}
\subsubsection*{a)}
\verb|opendir| returns a pointer to the first element of the open directory as an opaque data type.\\
Quote from Linux's Programmers Manual:\\
\say{The opendir function opens a directory stream corresponding to the directory name, and returns a pointer to the directory stream.  The stream is positioned at the first entry in the directory.}
\subsubsection*{b)}
It is, because \verb|opendir| and \verb|open| return similar information about multiple or one files: \\
\verb|opendir| returns information about filenames and their location, whereas \verb|open| returns a file descriptor, making it possible to read and write to a file. \\
Both functions prepare the usage of a directory / file while only giving basic information about them.
\subsection*{2)}
It's not, because \verb|readdir| returns \verb|NULL| when it reaches the end of a \verb|DIR*| from \verb|opendir|. The \verb|DIR*| may be subjected to corruption which may lead to \verb|readdir| to return \verb|NULL|. For example an application may call \verb|readdir| after \verb|closedir| on accident which results in undefined behavior.
\subsection*{3)}
- normal files (like user made files)\\
- device files (like some files under \verb|/dev/|)\\
- directory files (like link files) \\
- communication channels (like \verb|/dev/stdout|) \pagebreak
\subsection*{4)}
- Size \\
- Blocks used \\
- Type of file \\
- Linkcounter
\subsection*{5)}
\subsubsection*{a)}
A hard link stores information about the inode of a file.
\subsubsection*{b)}
A soft link stores information about the path of a file.
\subsection*{6)}
\say{-l: use a long listing format}, which also makes it display the sizes of each file, last written date and time, total bytes and permissions. \\
-R executes ls for each directory in the current directory, that is not \say{..}. \\
-r executes ls but lists in reversed order. \\
-1 executes ls but lists in single-collum format.
\end{document}