\documentclass[12pt, a4paper]{article}

\usepackage[english]{babel} 
\usepackage[T1]{fontenc}
\usepackage{amsfonts} 
\usepackage{setspace}
\usepackage{amsmath}
\usepackage{amssymb}
\usepackage{titling}
\usepackage{csquotes} % for \textquote{}
\usepackage{listings}
\usepackage{tikz}
\usetikzlibrary{arrows, automata, positioning}

\usepackage[
    left = \glqq{},% 
    right = \grqq{},% 
    leftsub = \glq{},% 
    rightsub = \grq{} %
]{dirtytalk}


\newcommand*{\qed}{\null\nobreak\hfill\ensuremath{\square}}
\newcommand*{\puffer}{\text{ }\text{ }\text{ }\text{ }}
\newcommand*{\gedanke}{\textbf{-- }}
\newcommand*{\gap}{\text{ }}
\newcommand*{\setDef}{\gap|\gap}
\newcommand*{\vor}{\textbf{Vor.:} \gap}
\newcommand*{\beh}{\textbf{Beh.:} \gap}
\newcommand*{\bew}{\textbf{Bew.:} \gap}
% Hab länger gebraucht um zu realisieren, dass das ne gute Idee wäre
\newcommand*{\R}{\mathbb R}


\pagestyle{plain}
\allowdisplaybreaks

\setlength{\droptitle}{-14em}
\setlength{\jot}{12pt}

\title{Operating Systems\\Assignment 3.2}
\author{Henri Heyden, Nike Pulow \\ \small stu240825, stu239549}
\date{}


\lstdefinestyle{code}{
    basicstyle=\ttfamily\footnotesize
}
\lstset{style = code}


\begin{document}
\maketitle
\subsubsection*{1.}
I don't know how to make cornered edges in a tikzpicture automata, but this should suffice. The numerics describe in which line we currently are, if there is an if or else block. Note that if \verb|fork()| returns a value subzero, then the further execution of this path will fail and this picture will not make sense.\\
\hspace*{-3cm}
\begin{tikzpicture}[auto, ->, >=stealth', node distance=3cm, semithick, block/.style={maximum size =1.5cm},  font=\small, align = center]
    \tikzstyle{every state}=[draw=none]
    \node (0) {l0};
    \node[right of = 0] (3) {l3};
    \node[right of = 3] (14) {l14};
    \node[right of = 14] (pW) {\verb|"Welcome"|};
    \node[right of = pW] (17) {l17};
    \node[right of = 17] (pF) {\verb|"Farewell"|};
    \node[right of = pF] (r0) {\verb|return(0);|};

    \node[above right = 1.732cm and 2cm of 17] (pB) {\verb|"Bye"|};
    \node[right of = pB] (r1) {\verb|return(0);|};

    
    \node[above right = 4cm and 2.3cm of 3] (4) {l4}; % tikz makes me fiddle with it more like with ...
    \node[right of = 4] (7) {l7};
    \node[right of = 7] (9) {l9};
    \node[right of = 9] (pH) {\verb|"Howdy"|};
    \node[right of = pH] (e0) {\verb|exit(0);|};

    \node[above right = 1.732cm and 2.3cm of 4] (pHi) {\verb|"Hi"|};
    \node[above right = 1.732cm and 2.3cm of pHi] (e1) {\verb|exit(0);|};

    \node[above right = 1.732cm and 2.3cm of 7] (pHe) {\verb|"Hello"|};
    \node[right of = pHe] (e2) {\verb|exit(0);|};

    \path (0) edge [] node {} (3);
    \path (3) edge [] node {\(\ne 0\)} (14);
    \path (14) edge [] node {} (pW);
    \path (pW) edge [] node {\(\ne 0\)} (17);
    \path (17) edge [] node {} (pF);
    \path (pF) edge [] node {} (r0);

    \path (17) edge [] node {\(=0\)} (pB);
    \path (pB) edge [] node {} (r1);

    \path (3) edge [] node {\(=0\)} (4);
    \path (4) edge [] node {\(\ne 0\)} (7);
    \path (7) edge [] node {\(\ne 0\)} (9);
    \path (9) edge [] node {} (pH);
    \path (pH) edge [] node {} (e0);

    \path (4) edge [] node {\(= 0\)} (pHi);
    \path (pHi) edge [] node {} (e1);

    \path (7) edge [] node {\(= 0\)} (pHe);
    \path (pHe) edge [] node {} (e2);

\end{tikzpicture}
\subsubsection*{2.}
\textbf{a)} False \\
\textbf{b)} True \\
\textbf{c)} False \\
\textbf{d)} False \\
\textbf{e)} False \\
\textbf{f)} True
\subsubsection*{3.}
TODO. Möglich das Diagramm davor zu rotieren, zu verschieben und einen Pfad als rote Pfeile zu markieren.
\end{document}